\documentclass[11pt,a4paper]{article}
\usepackage[utf8x]{inputenc}
\usepackage{textalpha}
\usepackage{amsmath}
\usepackage{graphicx}
\usepackage{bm}

\setcounter{MaxMatrixCols}{63}

\makeatletter
\renewcommand\paragraph{%
\@startsection{paragraph}{4}{0mm}%
{-\baselineskip}%
{.5\baselineskip}%
{\normalfont\normalsize\bfseries}}
\makeatother

\title{Ausarbeitung Kanalkodierung}
\author{Elisabeth Baudisch}

\begin{document}

\setlength{\parindent}{0em} 

\date{Februar 2019}
\maketitle

\begin{center}
Matrikelnummer: 3668504\\
Mail: e.baudisch@tu-dresden.de
\end{center}

\section{Aufgabenbeschreibung}

Wählen Sie eine Kodebeschreibung zur Untersuchung. 
\paragraph{Gewählte Beschreibung:} 

(n, l, dmin) = (63,31,7), mit \\ \textit{g(x)} = \textit{m$_{5}$m$_{11}$m$_{15}$m$_{21}$m$_{23}$m$_{31}$} = \textit{x$^{32}$} + \textit{x$^{16}$} + \textit{x$^{8}$} + \textit{x$^{4}$} + \textit{x$^{2}$} + \textit{x} + 1\\
\textit{h(x)} = \textit{x$^{31}$} + \textit{x$^{15}$} + \textit{x$^{7}$} + \textit{x$^{3}$} + \textit{x} + 1 \\

Wenden Sie eine Kombinationen von Dekodierungsalgorithmen auf die gewählte Kodebeschreibung an.

\paragraph{Gewählte Beschreibung:}

hard/soft reliability-based iterativer MLG (setzt Hn×n α -Anpassung voraus!)

\section{BCH-Code-Beschreibung}

Grundlage der Überlegung bildet ein (63,31,7)-BCH-Kode.
Da es sich um einen zyklischen Kode handelt, ergibt sich via zyklischer Verschiebung eine orthogonale Kontrollmatrix $H_{63 \times 63}$. In in Abbildung \ref{fig:HK_G_H} wird diese als Pixeldarstellung dargestellt.\\

\begin{figure}
	\includegraphics[width=\linewidth]{hmatrix.png}
	\caption{Zyklische Kontrollmatrix $H_{63 \times 63}$ \textit{h(x)} = \textit{x$^{31}$} + \textit{x$^{15}$} + \textit{x$^{7}$} + \textit{x$^{3}$} + \textit{x} + 1  in Pixeldarstellung. Schwarze Pixel = 1, ansonsten = 0}
	\label{fig:HK_G_H}
\end{figure}


Unter der Annahme, dass es sich bei der Beschreibung um einen primitiven BCH-Kode handelt, können folgende Kodeparameter abgeleitet werden: 

\textit{n$_{(max)}$} = 2\textit{$^{k_{1}}$} - 1, unter der Annahme \textit{n} = \textit{n$_{(max)}$}, dann 63 = 2\textit{$^{k_{1}}$} - 1, \textit{$k_{1}$} = 6 \\
\textit{k} = grad\textit{g(x)} = 32
mit \textit{d$_{min}$} = 7, daher benötigt der Kode \textit{d$_{min}$} - 1 = 6 aufeinanderfolgende Nullstellen

Die Zyklen der Elemente des Erweiterungskörpers mit \textit{n} = \textit{p} = 63 sind \\

$\alpha$$^{0}$ \\
$\alpha$$^{1}$, $\alpha$$^{2}$, $\alpha$$^{4}$, $\alpha$$^{8}$, $\alpha$$^{16}$, $\alpha$$^{32}$ \\
$\alpha$$^{3}$, $\alpha$$^{6}$, $\alpha$$^{12}$, $\alpha$$^{24}$, $\alpha$$^{48}$, $\alpha$$^{33}$ \\
$\alpha$$^{5}$, $\alpha$$^{10}$, $\alpha$$^{20}$, $\alpha$$^{40}$, $\alpha$$^{17}$, $\alpha$$^{34}$ \\
$\alpha$$^{7}$, $\alpha$$^{14}$, $\alpha$$^{28}$, $\alpha$$^{56}$, $\alpha$$^{49}$, $\alpha$$^{35}$ \\
$\alpha$$^{9}$, $\alpha$$^{18}$, $\alpha$$^{36}$ \\ $\alpha$$^{11}$, $\alpha$$^{22}$, $\alpha$$^{44}$,
$\alpha$$^{25}$, $\alpha$$^{50}$, $\alpha$$^{37}$ \\
$\alpha$$^{13}$, $\alpha$$^{26}$, $\alpha$$^{52}$, $\alpha$$^{41}$, $\alpha$$^{19}$, $\alpha$$^{38}$ \\
$\alpha$$^{15}$, $\alpha$$^{30}$, $\alpha$$^{60}$, $\alpha$$^{57}$, $\alpha$$^{51}$, $\alpha$$^{39}$ \\
$\alpha$$^{21}$, $\alpha$$^{42}$ \\
$\alpha$$^{23}$, $\alpha$$^{46}$, $\alpha$$^{29}$, $\alpha$$^{58}$, $\alpha$$^{53}$, $\alpha$$^{43}$ \\
$\alpha$$^{27}$, $\alpha$$^{54}$, $\alpha$$^{45}$ \\
$\alpha$$^{31}$, $\alpha$$^{62}$, $\alpha$$^{61}$, $\alpha$$^{59}$, $\alpha$$^{55}$, $\alpha$$^{47}$ \\

Die Zyklen der Minimalpolynome m$_{5}$ m$_{11}$ m$_{15}$ m$_{21}$ m$_{23}$ m$_{31}$ umfassen insgesamt 32 Elemente, entsprechend dem grad \textit{g(x)} = \textit{k} = 32. \\

Für die Bildung steht eine Menge an geeigneten Modularpolynomen zur Verfügung.

Auf Basis des Modularpolynoms, ergeben sich die Liste der Polynomrestwerte und über die Elemente der Erweiterungskörper können die relevanten  Minimalpolynome m0 bis m31 definiert werden.

% TODO: POLYNOMRESTWERTTABELLE
% TODO: Modularpolynom und enthaltene Minimalpolynome
% TODO: Evtl Aufeinanderfolgende Nullstellenfolgen



\section{LDPC-Encoder}
Der oben beschriebene BCH-Kode und entsprechend die aus ihm hervorgehende Kontrollmatrix bildet die Grundlage für einen LDPC-Code.

Da die zugrundeliegende Kontrollmatrix \textit{H} zyklisch ist, wird folgende Bildungsvorschrift angewendet.

\[ a_{j} = \sum_{j′∈K_{i}\backslash_{j}} a_{j′}  mod  2 , (j = l + i, i = 1, 2, ..., k)  \]

\section{Majority Logic-Decoder}
Hier kommen Bilder und Erkenntnisse über den Decoder-Vorgang und Ergebnisse



\end{document}